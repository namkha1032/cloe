% ICLR-like one-column LaTeX skeleton for Assignment 3
\documentclass{article}
\usepackage{authblk}
\usepackage{times}
\usepackage{graphicx}
\usepackage{amsmath}
\usepackage{booktabs}
\usepackage{hyperref}
\usepackage{natbib}

\title{Title Here: Deep Learning for Blood-Cell Classification}
\author[1]{Author Name}
\affil[1]{Institution, \texttt{email@domain.edu}}
\date{}

\begin{document}
\maketitle

\begin{abstract}
% 150-250 words summarizing problem, approach, and key results.
\end{abstract}

% Quick checklist for assignment requirements
\paragraph{Checklist}
\begin{itemize}
	\item Use CVPR or ICLR template (do not modify margins or fonts).
	\item Keep main text to 6 pages (references excluded).
	\item Include at least two baselines and an ablation study in Experiments.
\end{itemize}

\section{Introduction}
% Describe problem, dataset, motivation, and contributions.

\section{Related Work}
% Summarize past work and justify design choices.

\section{Method}
% Describe model architecture, inputs/outputs, training details, loss, optimizer.

\section{Experiments and Results}
% Dataset split, metrics, baselines, ablation study, tables and figures.
\subsection{Example Figures and Tables}
\begin{figure}[ht]
	\centering
	\includegraphics[width=0.8\linewidth]{figures/sample_figure.png}
	\caption{Sample figure placeholder — replace with your own images.}
	\label{fig:sample}
\end{figure}

\begin{table}[ht]
	\centering
	\begin{tabular}{lc}
		oprule
	Method & Accuracy (\%) \\
	\midrule
	Baseline A & 00.0 \\
	Baseline B & 00.0 \\
	Proposed & 00.0 \\
	\bottomrule
	\end{tabular}
	\caption{Example results table.}
	\label{tab:results}
\end{table}

\section{Discussion}
% Limitations, future work, concluding remarks.

\bibliographystyle{plain}
\bibliography{references}

\end{document}
